\documentclass[a4paper]{book}
\usepackage[times,inconsolata,hyper]{Rd}
\usepackage{makeidx}
\usepackage[utf8]{inputenc} % @SET ENCODING@
% \usepackage{graphicx} % @USE GRAPHICX@
\makeindex{}
\begin{document}
\chapter*{}
\begin{center}
{\textbf{\huge Package `animaltracker'}}
\par\bigskip{\large \today}
\end{center}
\begin{description}
\raggedright{}
\inputencoding{utf8}
\item[Title]\AsIs{Animal Tracker}
\item[Version]\AsIs{0.0.1.9000}
\item[Description]\AsIs{Import, visualize, and analyze GPS and accelerometer data for spatial-temporal tracking of animals (e.g., cows).}
\item[Depends]\AsIs{R (>= 3.3)}
\item[Imports]\AsIs{shiny (>= 1.2.0), xts (>= 0.11.2), leaflet (>= 2.0.2), dplyr (>= 0.7.5), sp (>= 1.3.1), ggplot2 (>= 3.1.0), scales (>= 1.0.0), tidyr (>= 0.8.2), sp (>= 1.3.1), rgdal (>= 1.3.6),  raster(>= 2.7.15), nabor(>= 0.5.0), elevatr (>= 0.2.0), geosphere (>= 1.5.7), RSQLite(>= 2.1.1)}
\item[License]\AsIs{GPL-3}
\item[Encoding]\AsIs{UTF-8}
\item[LazyData]\AsIs{true}
\item[RoxygenNote]\AsIs{6.1.1}
\end{description}
\Rdcontents{\R{} topics documented:}
\inputencoding{utf8}
\HeaderA{add\_to\_gitignore}{Add big files to a .gitignore file}{add.Rul.to.Rul.gitignore}
%
\begin{Description}\relax
Add big files to a .gitignore file
\end{Description}
%
\begin{Usage}
\begin{verbatim}
add_to_gitignore(data_dir)
\end{verbatim}
\end{Usage}
%
\begin{Arguments}
\begin{ldescription}
\item[\code{data\_dir}] directory of animal data files
\end{ldescription}
\end{Arguments}
\inputencoding{utf8}
\HeaderA{boxplot\_altitude}{Generates a boxplot to visualize the distribution of altitude by GPS.}{boxplot.Rul.altitude}
%
\begin{Description}\relax
Generates a boxplot to visualize the distribution of altitude
by GPS.
\end{Description}
%
\begin{Usage}
\begin{verbatim}
boxplot_altitude(rds_path)
\end{verbatim}
\end{Usage}
%
\begin{Arguments}
\begin{ldescription}
\item[\code{rds\_path}] Path of .rds animal data file to read in
\end{ldescription}
\end{Arguments}
%
\begin{Value}
overall boxplot of altitude by GPS
\end{Value}
\inputencoding{utf8}
\HeaderA{boxplot\_time\_unit}{Generates a boxplot to visualize the distribution of time between GPS measurements by GPS unit.}{boxplot.Rul.time.Rul.unit}
%
\begin{Description}\relax
Generates a boxplot to visualize the distribution of time between
GPS measurements by GPS unit.
\end{Description}
%
\begin{Usage}
\begin{verbatim}
boxplot_time_unit(rds_path)
\end{verbatim}
\end{Usage}
%
\begin{Arguments}
\begin{ldescription}
\item[\code{rds\_path}] Path of .rds animal data file to read in
\end{ldescription}
\end{Arguments}
%
\begin{Value}
distribution of time between GPS measurements by GPS unit, as a boxplot
\end{Value}
\inputencoding{utf8}
\HeaderA{clean\_batch}{Cleans a directory of animal data files and stores them in .rds files}{clean.Rul.batch}
%
\begin{Description}\relax
Cleans a directory of animal data files and stores them in .rds files
\end{Description}
%
\begin{Usage}
\begin{verbatim}
clean_batch(data_dir)
\end{verbatim}
\end{Usage}
%
\begin{Arguments}
\begin{ldescription}
\item[\code{data\_dir}] location of animal data files, in list format
\end{ldescription}
\end{Arguments}
%
\begin{Value}
df of metadata for animal file directory
\end{Value}
\inputencoding{utf8}
\HeaderA{clean\_df}{Clean animal data frame}{clean.Rul.df}
%
\begin{Description}\relax
Clean animal data frame
\end{Description}
%
\begin{Usage}
\begin{verbatim}
clean_df(df, ani_id, gps_id)
\end{verbatim}
\end{Usage}
%
\begin{Arguments}
\begin{ldescription}
\item[\code{df}] raw input data frame

\item[\code{ani\_id}] animal ID (from meta)

\item[\code{gps\_id}] GPS ID (from meta)
\end{ldescription}
\end{Arguments}
%
\begin{Value}
cleaned data frame
\end{Value}
\inputencoding{utf8}
\HeaderA{clean\_export\_files}{Cleans all animal GPS datasets in a chosen directory and exports them as a single .rds file}{clean.Rul.export.Rul.files}
%
\begin{Description}\relax
Cleans all animal GPS datasets in a chosen directory and exports them as a single .rds file
\end{Description}
%
\begin{Usage}
\begin{verbatim}
clean_export_files(data_dir, out_path, processed_dir = "data/processed")
\end{verbatim}
\end{Usage}
%
\begin{Arguments}
\begin{ldescription}
\item[\code{data\_dir}] directory of GPS tracking files (in csv)

\item[\code{out\_path}] name of output file, must end in .rds

\item[\code{processed\_dir}] directory of processed GPS datasets
\end{ldescription}
\end{Arguments}
\inputencoding{utf8}
\HeaderA{export\_animal\_elevation}{Export modeled elevation data from existing animal data file}{export.Rul.animal.Rul.elevation}
%
\begin{Description}\relax
Export modeled elevation data from existing animal data file
\end{Description}
%
\begin{Usage}
\begin{verbatim}
export_animal_elevation(rds_path, out_path)
\end{verbatim}
\end{Usage}
%
\begin{Arguments}
\begin{ldescription}
\item[\code{rds\_path}] animal tracking data file to model elevation from

\item[\code{out\_path}] exported file path
\end{ldescription}
\end{Arguments}
%
\begin{Value}
list of data frames with gps data augmented by elevation
\end{Value}
\inputencoding{utf8}
\HeaderA{get\_data\_from\_meta}{Get animal data set from specified meta}{get.Rul.data.Rul.from.Rul.meta}
%
\begin{Description}\relax
Get animal data set from specified meta
\end{Description}
%
\begin{Usage}
\begin{verbatim}
get_data_from_meta(meta_df, min_date, max_date, min_time, max_time)
\end{verbatim}
\end{Usage}
%
\begin{Arguments}
\begin{ldescription}
\item[\code{meta\_df}] data frame of specified meta

\item[\code{min\_date}] minimum date specified by user

\item[\code{max\_date}] maximum date specified by user

\item[\code{min\_time}] minimum time specified by user

\item[\code{max\_time}] maximum time specified by user
\end{ldescription}
\end{Arguments}
\inputencoding{utf8}
\HeaderA{get\_elevation}{Retrieve and save high resolution elevation data for the region of analysis from the internet}{get.Rul.elevation}
%
\begin{Description}\relax
Retrieve and save high resolution elevation data for the region of analysis from the internet
\end{Description}
%
\begin{Usage}
\begin{verbatim}
get_elevation(latmin, latmax, lonmin, lonmax, out_dir, zoom = 12,
  zone = 11)
\end{verbatim}
\end{Usage}
%
\begin{Arguments}
\begin{ldescription}
\item[\code{latmin}] minimum latitude for bounding box (degrees)

\item[\code{latmax}] maximum latitude for bounding box (degrees)

\item[\code{lonmin}] minimum longitude for bounding box (degrees)

\item[\code{lonmax}] maximum longitude for bounding box (degrees)

\item[\code{out\_dir}] folder path to save the elevation data

\item[\code{zoom}] level of zoom, defaults to 12

\item[\code{zone}] geographic zone, defaults to 11
\end{ldescription}
\end{Arguments}
%
\begin{Value}
elevation data as spatial points
\end{Value}
\inputencoding{utf8}
\HeaderA{get\_file\_meta}{Generate metadata for a directory of animal data files}{get.Rul.file.Rul.meta}
%
\begin{Description}\relax
Generate metadata for a directory of animal data files
\end{Description}
%
\begin{Usage}
\begin{verbatim}
get_file_meta(data_dir)
\end{verbatim}
\end{Usage}
%
\begin{Arguments}
\begin{ldescription}
\item[\code{data\_dir}] directory of animal data files
\end{ldescription}
\end{Arguments}
%
\begin{Value}
list of data info as a list of animal IDs and GPS units
\end{Value}
\inputencoding{utf8}
\HeaderA{get\_meta}{Generate metadata for an animal data frame - filename, site, date min/max, animals, min/max lat/longitude, storage location}{get.Rul.meta}
%
\begin{Description}\relax
Generate metadata for an animal data frame -
filename, site, date min/max, animals, min/max lat/longitude, storage location
\end{Description}
%
\begin{Usage}
\begin{verbatim}
get_meta(df, file_id, file_name, site, ani_id, storage_loc)
\end{verbatim}
\end{Usage}
%
\begin{Arguments}
\begin{ldescription}
\item[\code{df}] clean animal data frame

\item[\code{file\_id}] ID number of .csv source of animal data frame

\item[\code{file\_name}] .csv source of animal data frame

\item[\code{ani\_id}] ID of animal found in data frame

\item[\code{storage\_loc}] .rds storage location of animal data frame
\end{ldescription}
\end{Arguments}
%
\begin{Value}
df of metadata for animal data frame
\end{Value}
\inputencoding{utf8}
\HeaderA{histogram\_animal\_elevation}{Generate a histogram of the distribution of modeled elevation - measured altitude}{histogram.Rul.animal.Rul.elevation}
%
\begin{Description}\relax
Generate a histogram of the distribution of modeled elevation - measured altitude
\end{Description}
%
\begin{Usage}
\begin{verbatim}
histogram_animal_elevation(csv_path)
\end{verbatim}
\end{Usage}
%
\begin{Arguments}
\begin{ldescription}
\item[\code{csv\_path}] path of csv GPS data to model elevation from
\end{ldescription}
\end{Arguments}
%
\begin{Value}
histogram of the distribution of modeled elevation - measured altitude
\end{Value}
\inputencoding{utf8}
\HeaderA{histogram\_time}{Generates a histogram to visualize the distribution of time between GPS measurements.}{histogram.Rul.time}
%
\begin{Description}\relax
Generates a histogram to visualize the distribution of time
between GPS measurements.
\end{Description}
%
\begin{Usage}
\begin{verbatim}
histogram_time(rds_path)
\end{verbatim}
\end{Usage}
%
\begin{Arguments}
\begin{ldescription}
\item[\code{rds\_path}] Path of .rds cow data file to read in
\end{ldescription}
\end{Arguments}
%
\begin{Value}
distribution of time between GPS measurements, as a histogram
\end{Value}
\inputencoding{utf8}
\HeaderA{histogram\_time\_unit}{Generates a histogram to visualize the distribution of time between GPS measurements by GPS unit.}{histogram.Rul.time.Rul.unit}
%
\begin{Description}\relax
Generates a histogram to visualize the distribution of time between
GPS measurements by GPS unit.
\end{Description}
%
\begin{Usage}
\begin{verbatim}
histogram_time_unit(rds_path)
\end{verbatim}
\end{Usage}
%
\begin{Arguments}
\begin{ldescription}
\item[\code{rds\_path}] Path of .rds animal data file to read in
\end{ldescription}
\end{Arguments}
%
\begin{Value}
distribution of time between GPS measurements by GPS unit, as a histogram
\end{Value}
\inputencoding{utf8}
\HeaderA{model\_animal\_elevation}{Model elevation from GPS data (provided csv)}{model.Rul.animal.Rul.elevation}
%
\begin{Description}\relax
Model elevation from GPS data (provided csv)
\end{Description}
%
\begin{Usage}
\begin{verbatim}
model_animal_elevation(csv_path)
\end{verbatim}
\end{Usage}
%
\begin{Arguments}
\begin{ldescription}
\item[\code{csv\_path}] path of csv GPS data
\end{ldescription}
\end{Arguments}
%
\begin{Value}
modeled elevation data
\end{Value}
\inputencoding{utf8}
\HeaderA{qqplot\_time}{Generates a QQ plot to show the distribution of time between GPS measurements.}{qqplot.Rul.time}
%
\begin{Description}\relax
Generates a QQ plot to show the distribution of time between GPS measurements.
\end{Description}
%
\begin{Usage}
\begin{verbatim}
qqplot_time(rds_path)
\end{verbatim}
\end{Usage}
%
\begin{Arguments}
\begin{ldescription}
\item[\code{rds\_path}] Path of .rds animal data file to read in
\end{ldescription}
\end{Arguments}
%
\begin{Value}
quantile-quantile plot to show distribution of time between GPS measurements
\end{Value}
\inputencoding{utf8}
\HeaderA{quantile\_time}{Determines the GPS measurement time value difference values roughly corresponding to quantiles with .05 intervals.}{quantile.Rul.time}
%
\begin{Description}\relax
Determines the GPS measurement time value difference values
roughly corresponding to quantiles with .05 intervals.
\end{Description}
%
\begin{Usage}
\begin{verbatim}
quantile_time(rds_path)
\end{verbatim}
\end{Usage}
%
\begin{Arguments}
\begin{ldescription}
\item[\code{rds\_path}] Path of .rds animal data file to read in
\end{ldescription}
\end{Arguments}
%
\begin{Value}
approximate time difference values corresponding to quantiles (.05 intervals)
\end{Value}
\inputencoding{utf8}
\HeaderA{run\_shiny\_animaltracker}{You can run the animaltracker Shiny app by calling this function.}{run.Rul.shiny.Rul.animaltracker}
%
\begin{Description}\relax
You can run the animaltracker Shiny app by calling this function.
\end{Description}
%
\begin{Usage}
\begin{verbatim}
run_shiny_animaltracker()
\end{verbatim}
\end{Usage}
%
\begin{Arguments}
\begin{ldescription}
\item[\code{rds\_path}] Path of Animal data file to input
\end{ldescription}
\end{Arguments}
\inputencoding{utf8}
\HeaderA{save\_meta}{Save metadata to a data frame and return it}{save.Rul.meta}
%
\begin{Description}\relax
Save metadata to a data frame and return it
\end{Description}
%
\begin{Usage}
\begin{verbatim}
save_meta(meta_df, file_meta)
\end{verbatim}
\end{Usage}
%
\begin{Arguments}
\begin{ldescription}
\item[\code{meta\_df}] the data frame to store metadata in

\item[\code{file\_meta}] meta for a .csv file generated by get\_meta
\end{ldescription}
\end{Arguments}
\inputencoding{utf8}
\HeaderA{summarize\_unit}{Summarize by GPS unit}{summarize.Rul.unit}
%
\begin{Description}\relax
Summarize by GPS unit
\end{Description}
%
\begin{Usage}
\begin{verbatim}
summarize_unit(rds_path)
\end{verbatim}
\end{Usage}
%
\begin{Arguments}
\begin{ldescription}
\item[\code{rds\_path}] Path of .rds cow data file to read in
\end{ldescription}
\end{Arguments}
%
\begin{Value}
summary statistics for animals by GPS unit
\end{Value}
\printindex{}
\end{document}
